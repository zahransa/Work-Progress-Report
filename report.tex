\documentclass[a4paper]{article}

\usepackage[english]{babel}
\usepackage[utf8]{inputenc}
\usepackage{fullpage}
\usepackage{amsmath}
\usepackage{graphicx}
\usepackage[colorinlistoftodos]{todonotes}
\usepackage{hyperref}
\usepackage{amssymb}
\usepackage{outline} \usepackage{pmgraph} \usepackage[normalem]{ulem}
\usepackage{graphicx} \usepackage{verbatim}
% \usepackage{minted} % need `-shell-escape' argument for local compile

\title{
    \vspace*{1in}
    %\includegraphics[width=2.75in]{figures/zhenglab-logo} \\
    \vspace*{1.2in}
    \textbf{\huge Work Progress Report}
    \vspace{0.2in}
}

\author{Saeed ZAHRAN \\
    \vspace*{0.5in} \\
  %  \textbf{VISION@OUC} \\
    \vspace*{1in}
}

\date{\today}


\begin{document}

\maketitle
\setcounter{page}{0}
\thispagestyle{empty}
\newpage


\section{Research problem}

What are you researching? Talk about your motivation.

\section{Research approach}

Write down the approach you use in your research study, this will help you when writing the ``Method'' part in the paper. You may reference some papers like~\cite{isola2017image}.


\section{Research progress}

How much work you have done before this work? Write down your previous work for this project, so people can quickly figure out where you are in your road map now. Yeah, maybe you walked through a long and hard road.

Maybe you have got many data and results, list some necessary here, like Table~\ref{tab:result} shows.

\begin{table}[hb]
    \centering
    \begin{tabular}{c|c}
        \hline \\
        name & value \\
        \hline \\
        a & 0 \\
        \hline \\
        b & 1 \\
        \hline
    \end{tabular}
    \caption{Your experiment result.}
    \label{tab:result}
\end{table}


\section{Progress in this week}

List what you have done in this week in detail.

For example, maybe you performed some experiments this week. The following are the steps you took:

\begin{description}
\item [Step 1]
Got up to welcome a new day.
\item[Step 2]
Opened your computer to start a new day's work.
\item[Step 3]
Got stuck with a very hard problem, like $e^{i \pi} + 1 = 0$.
\item[Step 4]
You searched online and realized some useful information like Figure~\ref{fig:google} shows. You asked other people for help and got the things done luckily.
\end{description}

\begin{figure}[hb]
    \centering
    \includegraphics{figures/google}
    \caption{Search online to get some useful information.}
    \label{fig:google}
\end{figure}

But there are still some problems confusing you, then you need to keep calm and carry on.

\section{Plan}

\begin{tabular}{rl}
	\textbf{Objective:} & XXXX \\
    \textbf{Deadline:} & XXXX 
\end{tabular}

\begin{description}
    \item[\normalfont 2018.05.07---2018.05.14] Do something.
    \item[\normalfont 2018.05.15---2018.05.22] Do something else.
    \item[\normalfont 2018.05.23---2018.05.31] Do a lot lot lot lot lot lot lot lot lot lot lot lot of things.
\end{description}

% If you don't cite any references, please comment the following two lines
\bibliographystyle{ieee}
\bibliography{ref.bib}

\end{document}